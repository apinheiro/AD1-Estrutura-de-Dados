\documentclass[a4paper,11pt]{article}
\usepackage[utf8]{inputenc}
\usepackage[brazil]{babel}
\usepackage[portuguese,linesnumbered,ruled, vlined]{algorithm2e}
\usepackage{amsmath}


\title{AD1 Estrutura de Dados - 2018.2}
\author{André Luiz Rodrigues Pinheiro - 18113050170}

\begin{document}

\maketitle

\section*{Respostas:}

\subsection*{Questão 3:}

\begin{algorithm}
\caption{Agrupar\_Elementos(V)}
\Entrada{Um vetor contendo n elementos}
i := 0\\
\Para{i $\leq$ n}{
 j := V[i] / 3 \textit{\% j inteiro} \\
 S[j] := S[j] + 1 \\
 i := i+ 1}
 \Para{k \Ate n}{
   \textbf{imprima} S[i],"elementos numa faixa de ",(3*i)+1,"até",3*(i+1)
 }
\endpara
\end{algorithm}

Este algoritmo possui complexidade O(n).

\subsection*{Questão 8:}

\begin{center}
\begin{tabular}{cccccccccc}
  10 & 9 & 8 & 7 & 6 & 5 & 4 & 3& 2 & 1* \\
  1 & 10 & 9 & 8 & 7 & 6 & 5 & 4 & 3& 2*  \\
  1 & 2 & 10 & 9 & 8 & 7 & 6 & 5 & 4 & 3*  \\
  1 & 2 & 3 & 10 & 9 & 8 & 7 & 6 & 5 & 4* \\
  1 & 2 & 3 & 4 & 10 & 9 & 8 & 7 & 6 & 5* \\
  1 & 2 & 3 & 4 & 5 & 10 & 9 & 8 & 7 & 6* \\
  1 & 2 & 3 & 4 & 5 & 6 & 10 & 9 & 8 & 7* \\
  1 & 2 & 3 & 4 & 5 & 6 & 7 & 10 & 9 & 8* \\
  1 & 2 & 3 & 4 & 5 & 6 & 7 & 8 & 10 & 9* \\
  1 & 2 & 3 & 4 & 5 & 6 & 7 & 8 & 9 & 10* \\
  1 & 2 & 3 & 4 & 5 & 6 & 7 & 8 & 9 & 10
\end{tabular}
\end{center}
Na primeira linha há 9 comparações, na segunda são 8 e assim por diante. O total de comparações é de:

\[ \sum_{i=1}^9 n = 1 + 2 + \cdots + n = n\frac{n+1}{2}\]

O$\left(\dfrac{n(n+1)}{2}\right)$ = O($n^2$). O total de comparações é de $\dfrac{9(10)}{2} = \dfrac{90}{2} = 45$ comparações.

\subsection*{Questão 9:}
\end{document}
