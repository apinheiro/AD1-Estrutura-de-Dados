\documentclass[a4paper,11pt]{article}
\usepackage[utf8]{inputenc}
\usepackage[brazil]{babel}
\usepackage[portuguese,linesnumbered,ruled, vlined]{algorithm2e}
\usepackage{amsmath}


\title{AD1 Estrutura de Dados - 2018.2}
\author{André Luiz Rodrigues Pinheiro - 18113050170}

\begin{document}

\maketitle

\section*{Respostas:}

\subsection*{Questão 1:}

\subsection*{Questão 2:}

\subsection*{Questão 3:}

\begin{algorithm}
\caption{Agrupar\_Elementos(V)}
\Entrada{Um vetor contendo n elementos}
i $\leftarrow$ 0\\
\Para{i $\leq$ n}{
 j $\leftarrow$ V[i] / 3 \textit{\% j inteiro} \\
 S[j] $\leftarrow$ S[j] + 1 \\
 i $\leftarrow$ i+ 1}
 \Para{k \Ate n}{
   \textbf{imprima} S[i],"elementos numa faixa de ",(3*i)+1,"até",3*(i+1)
 }
\end{algorithm}

Este algoritmo possui complexidade O(n).

\subsection*{Questão 4:}

\subsection*{Questão 5:}

\subsection*{Questão 6:}

\subsection*{Questão 7:}

\subsection*{Questão 8:}

\begin{center}
\begin{tabular}{cccccccccc}
  10 & 9 & 8 & 7 & 6 & 5 & 4 & 3& 2 & 1* \\
  1 & 10 & 9 & 8 & 7 & 6 & 5 & 4 & 3& 2*  \\
  1 & 2 & 10 & 9 & 8 & 7 & 6 & 5 & 4 & 3*  \\
  1 & 2 & 3 & 10 & 9 & 8 & 7 & 6 & 5 & 4* \\
  1 & 2 & 3 & 4 & 10 & 9 & 8 & 7 & 6 & 5* \\
  1 & 2 & 3 & 4 & 5 & 10 & 9 & 8 & 7 & 6* \\
  1 & 2 & 3 & 4 & 5 & 6 & 10 & 9 & 8 & 7* \\
  1 & 2 & 3 & 4 & 5 & 6 & 7 & 10 & 9 & 8* \\
  1 & 2 & 3 & 4 & 5 & 6 & 7 & 8 & 10 & 9* \\
  1 & 2 & 3 & 4 & 5 & 6 & 7 & 8 & 9 & 10* \\
  1 & 2 & 3 & 4 & 5 & 6 & 7 & 8 & 9 & 10
\end{tabular}
\end{center}

Na primeira linha há 9 comparações, na segunda são 8 e assim por diante. O total de comparações é de:

\[ \sum_{i=1}^9 n = 1 + 2 + \cdots + n = n\frac{n+1}{2}\]

A complexidade é  O$\left(\dfrac{n(n+1)}{2}\right)$ = O($n^2$). O total de comparações é de $\dfrac{9(10)}{2} = \dfrac{90}{2} = 45$ comparações.

\subsection*{Questão 9:}

\begin{center}
\begin{tabular}{cccccccccc}
10* & 9 & 8 & 7 & 6 & 5 & 4 & 3& 2 & 1 \\
10 & 9* & 8 & 7 & 6 & 5 & 4 & 3& 2 & 1 \\
9* & 10 & 8 & 7 & 6 & 5 & 4 & 3& 2 & 1 \\
9 & 10 & 8* & 7 & 6 & 5 & 4 & 3& 2 & 1 \\
9 & 8* & 10 & 7 & 6 & 5 & 4 & 3& 2 & 1 \\
8* & 9 & 10 & 7 & 6 & 5 & 4 & 3& 2 & 1 \\
8 & 9 & 10 & 7* & 6 & 5 & 4 & 3& 2 & 1 \\
8 & 9 & 7* & 10 & 6 & 5 & 4 & 3& 2 & 1 \\
8 & 7* & 9 & 10 & 6 & 5 & 4 & 3& 2 & 1 \\
7* & 8 & 9 & 10 & 6 & 5 & 4 & 3& 2 & 1 \\
7 & 8 & 9 & 10 & 6* & 5 & 4 & 3& 2 & 1 \\
7 & 8 & 9 & 6* & 10 & 5 & 4 & 3& 2 & 1 \\
7 & 8 & 6* & 9 & 10 & 5 & 4 & 3& 2 & 1 \\
7 & 6* & 8 & 9 & 10 & 5 & 4 & 3& 2 & 1 \\
6* & 7 & 8 & 9 & 10 & 5 & 4 & 3& 2 & 1 \\
6 & 7 & 8 & 9 & 10 & 5* & 4 & 3& 2 & 1 \\
6 & 7 & 8 & 9 & 5* & 10 & 4 & 3& 2 & 1 \\
6 & 7 & 8 & 5* & 9 & 10 & 4 & 3& 2 & 1 \\  
6 & 7 & 5* & 8 & 9 & 10 & 4 & 3& 2 & 1 \\
6 & 5* & 7 & 8 & 9 & 10 & 4 & 3& 2 & 1 \\
5* & 6 & 7 & 8 & 9 & 10 & 4 & 3& 2 & 1 \\
5 & 6 & 7 & 8 & 9 & 10 & 4* & 3& 2 & 1 \\
5 & 6 & 7 & 8 & 9 & 4* & 10 & 3& 2 & 1 \\
5 & 6 & 7 & 8 & 4* & 9 & 10 & 3& 2 & 1 \\
5 & 6 & 7 & 4* & 8 & 9 & 10 & 3& 2 & 1 \\
5 & 6 & 4* & 7 & 8 & 9 & 10 & 3& 2 & 1 \\
5 & 4* & 6 & 7 & 8 & 9 & 10 & 3& 2 & 1 \\
4* & 5 & 6 & 7 & 8 & 9 & 10 & 3& 2 & 1 \\
4 & 5 & 6 & 7 & 8 & 9 & 10 & 3* & 2 & 1 \\
4 & 5 & 6 & 7 & 8 & 9 & 3* & 10 & 2 & 1 \\
4 & 5 & 6 & 7 & 8 & 3* & 9 & 10 & 2 & 1 \\
4 & 5 & 6 & 7 & 3* & 8 & 9 & 10 & 2 & 1 \\
4 & 5 & 6 & 3* & 7 & 8 & 9 & 10 & 2 & 1 \\
4 & 5 & 3* & 6 & 7 & 8 & 9 & 10 & 2 & 1 \\
4 & 3* & 5 & 6 & 7 & 8 & 9 & 10 & 2 & 1 \\
\end{tabular}
\begin{tabular}{cccccccccc}
3* & 4 & 5 & 6 & 7 & 8 & 9 & 10 & 2 & 1 \\
3 & 4 & 5 & 6 & 7 & 8 & 9 & 10 & 2* & 1 \\
3 & 4 & 5 & 6 & 7 & 8 & 9 & 2* & 10 & 1 \\
3 & 4 & 5 & 6 & 7 & 8 & 2* & 9 & 10 & 1 \\
3 & 4 & 5 & 6 & 7 & 2* & 8 & 9 & 10 & 1 \\
3 & 4 & 5 & 6 & 2* & 7 & 8 & 9 & 10 & 1 \\
3 & 4 & 5 & 2* & 6 & 7 & 8 & 9 & 10 & 1 \\
3 & 4 & 2* & 5 & 6 & 7 & 8 & 9 & 10 & 1 \\
3 & 2* & 4 & 5 & 6 & 7 & 8 & 9 & 10 & 1 \\
2* & 3 & 4 & 5 & 6 & 7 & 8 & 9 & 10 & 1 \\
2 & 3 & 4 & 5 & 6 & 7 & 8 & 9 & 10 & 1* \\
2 & 3 & 4 & 5 & 6 & 7 & 8 & 9 & 1* & 10 \\
2 & 3 & 4 & 5 & 6 & 7 & 8 & 1* & 9 & 10 \\
2 & 3 & 4 & 5 & 6 & 7 & 1* & 8 & 9 & 10 \\
2 & 3 & 4 & 5 & 6 & 1* & 7 & 8 & 9 & 10 \\
2 & 3 & 4 & 5 & 1* & 6 & 7 & 8 & 9 & 10 \\
2 & 3 & 4 & 1* & 5 & 6 & 7 & 8 & 9 & 10 \\
2 & 3 & 1* & 4 & 5 & 6 & 7 & 8 & 9 & 10 \\
2 & 1* & 3 & 4 & 5 & 6 & 7 & 8 & 9 & 10 \\
1* & 2 & 3 & 4 & 5 & 6 & 7 & 8 & 9 & 10 \\
1 & 2 & 3 & 4 & 5 & 6 & 7 & 8 & 9 & 10
\end{tabular}
\end{center}

O número de comparações aumenta em PA de razão 1 para cada número comparado. 

Logo, o total de comparações feitas são:

\[ \sum_{i=1}^9 n = n \frac{n+1}{2} = 45\]


\subsection*{Questão 10:}
\end{document}
